\chapter{実験装置}
\section{プラズマチャンバ}
本研究で用いたプラズマチャンバの外観を図\ref{fig:chamber-picture}に,チャンバの構造を簡単にした図を図\ref{fig:chamber-simple}に示す.
プラズマチャンバは主にICF-152規格のスレンレス製六方クロス管が用いられている.
(a)流量調節バルブは水素ボンベに繋がっており,このバルブでチャンバ内の水素の圧力を調節することができる.
実験前には,バルブを閉じ切って圧力が$10^{-6}$ Torr程度に下がっていることを確認した後,バルブを少しずつ開いて$10^{-3}$ Torr程度になるまで水素を注入した.
(b)圧力計(PFEIFFER VACUUM PKR251)が取り付けられており,チャンバ内の圧力を調べることができる.
(c)の排気口はロータリーポンプとターボ分子ポンプに繋がっている.
(d)カソードには直径0.5 mmのタングステンフィラメントを使用し,(e)アノードとして直径50 mmの平板電極を使用した.
カソードの両端は(f)直流電源(菊水電子工業,REGULATED DC POWER SUPPLY)に繋ぎ,電流は20 A程度になるように調整することで,フィラメントから熱電子を放出させた.
アノードとカソードは別の(g)直流電源(高砂製作所, GP 0110-3)に繋がれている.
アノード,カソード間の電流は2.5 V程度になるように調節した.
(h)は石英窓であり,ここに(i)コリメータを取り付けた(j)光ファイバーを繋ぐことで,プラズマの光を光学系へ取り込んでいる.

\section{分光器}
分光器の概略図を図\ref{fig:spectrometer-picture}に示す.
プラズマからの光は,コリメータを用いて集光したものを(j)光ファイバーでツェルニ・ターナー型分光器へ入射させている.
分光器入り口には(k)スリットを設置していて,その幅は50 µmである.
スリットを通過した光は(l)コリメートミラーで平行光となり,(m)回折格子に入射する.
この回折格子は有効幅が10 cm, 刻線数は2400 本/mmである.
回折格子で反射された光は波長ごとに広がり,(n)ミラーで反射した後(o)CCDカメラで検出される.
CCDカメラの仕様\cite{CCD-spec}を表\ref{table:CCD-spec}に示す.
回折格子からの光はCCDカメラで一度に計測できる幅を超えて広がるので,回折格子を(p)ステッピングモーターで回転させることで,広い波長範囲の計測を行えるようになっている.
また,ステッピングモーターおよびCCDカメラを(q)Pythonのプログラムで制御することで,複数の波長範囲に対応する画像を自動的に撮影することができる.