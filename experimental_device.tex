\chapter{実験装置}
\section{プラズマチャンバ}
本研究で用いたプラズマチャンバの外観を図\ref{fig:chamber-picture}に,チャンバの構造を簡単にした図を図\ref{fig:chamber-simple}に示す.
プラズマチャンバは主にICF-152規格のスレンレス製六方クロス管が用いられている.
流量調節バルブは水素ボンベに繋がっており,このバルブでチャンバ内の水素の圧力を調節することができる.
圧力計(PFEIFFER VACUUM PKR251)が取り付けられており,チャンバ内の圧力を調べることができる.
排気口はロータリーポンプ(アネスト岩田,ISP-250B-SV)とターボ分子ポンプ(大阪真空機器製作所,TH162CA)に繋げた.
カソードには直径0.5 mmのタングステンフィラメントを使用し,アノードとして直径50 mmの平板電極を使用した.
カソードの両端は直流電源(菊水電子工業,REGULATED DC POWER SUPPLY)に繋ぎ,電流は20 A程度になるように調整することで,フィラメントから熱電子を放出させた.
アノードとカソードは別の直流電源(高砂製作所, GP 0110-3)に繋げた.
チャンバには石英窓が取り付けられており,ここにコリメータを付けた光ファイバーを繋ぐことで,プラズマの光を光学系へ取り込んだ.

\section{分光器}
分光器の概略図を図\ref{fig:spectrometer-picture}に示す.
プラズマからの光は,コリメータを用いて集光したものを光ファイバーでツェルニ・ターナー型分光器(堀場製作所,HR640)へ入射させている.
分光器入り口にはスリットを設置していて,その幅は50 µmである.
スリットを通過した光はコリメートミラーで平行光となり,回折格子に入射する.
この回折格子は有効幅が10 cm, 刻線数は2400 本/mmである.
回折格子で反射した光は,ミラーで反射した後CCDカメラ(Finger Lakes Instrumentation,ML1109)で検出される.
CCDカメラの仕様\cite{CCD-spec}を表\ref{table:CCD-spec}に示す.
CCDカメラが一度に計測できる波長範囲は8.8 nmだが,回折格子からの光はこの幅を超えて広がるので,回折格子をステッピングモーターで回転させることで,広い波長範囲の計測を行えるようになっている.
また,ステッピングモーターおよびCCDカメラをPythonのプログラムで制御することで,複数の波長範囲に対応する画像を自動的に撮影することができる.