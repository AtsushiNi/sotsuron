\chapter{緒言}
\begin{comment}
新たな発電方法として,核融合発電が注目されている.
核融合発電とは重水素と三重水素の核融合反応によって生じるエネルギーを発電に利用する方法である.
核融合発電のメリットとして,海水から燃料を作り出せることや,二酸化炭素を排出しないこと,暴走のリスクが低く高レベルの核廃棄物を出さないことなどが挙げられる\cite{monbu-kagakusyou}.
しかし,実現までの技術的課題は多く,そのうちの一つにプラズマの閉じ込め性能の問題がある.
閉じ込め領域内の水素イオンは,中性粒子となってプラズマの周辺領域へと拡散してしまう.
\end{comment}
新たな発電方法として,核融合プラズマを用いた発電が注目されている.
核融合プラズマは,高温・高密度の水素プラズマを磁場によって閉じ込めたものである.
しかし閉じ込め領域の粒子は周辺領域へと拡散する.
拡散したイオンや中性粒子が炉壁やプラズマの周辺機器であるダイバータに衝突すると,そこから水素原子分子が放出される.
水素原子分子はプラズマの中心領域へと戻り,電離して水素イオンになる.
これらの過程は水素リサイクリングと呼ばれ,閉じ込め性能に大きく影響する\cite{hiramatsu}.
この過程は多くの素過程を含み,その中でも分子活性化再結合はダイバータの損耗に影響があるとされる重要な素過程である.
そして,分子活性化再結合の反応係数は水素分子基底準位の振動・回転状態に大きく依存することが分かっている.
石原らによる先行研究では,大型ヘリカル装置(LHD)のプラズマ周辺領域における基底準位の振動・回転状態占有数分布が2温度のボルツマン分布の和でよく近似できることが示された.
しかし,このような占有数分布になる理由については炉壁との相互作用であると考えられてはいるが詳しくは解明されていない.
そこで本研究では,LHDとは大きく異なるプラズマである,タングステンの熱陰極を用いた水素プラズマにおけるFulcher-α帯の発光スペクトルを計測し,水素分子基底準位の振動・回転状態占有率を求めた.
またその結果をLHDでの研究結果と比較した.

\begin{comment}
案2 2温度の理由
基底準位の振動・回転状態占有率は色々な反応に影響する.
例えば〇〇.他には〇〇.cite1.1807810.pdf
占有数は定回転量子数域では1温度,高回転量子数域まで含めると2温度のボルツマン分布で近似できることが報告されているcite
ところがその理由は壁表面との相互作用が原因とされているものの,詳しいことはわかっていないcite
さまざまな条件で計測することで
\end{comment}