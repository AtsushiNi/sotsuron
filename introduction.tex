\chapter{緒言}
新エネルギーとして核融合発電が注目されている背景から,水素や重水素のプラズマ中の状態や,プラズマと炉壁との関係を調べる研究が,数多く行われている.
そのうちの一つに,透過プローブを用いて水素原子の入射流束を測定する研究がある\cite{yun}.
これは,プラズマから飛来する粒子のうち水素原子のみがPdCu膜を容易に透過できることを利用し,プラズマ中の水素原子密度を求めるものである.
しかし,透過プローブの性能はPdCu膜表面の状態に影響されることがわかっており,またプラズマ中水素分子の振動・回転状態によっても変化する可能性がある.
また別の研究では,プラズマ中に置いた炭素やタングステン等の資料に重水素がどれほど吸収されるかを調べている\cite{kuzmin}.
この研究においても,水素分子の振動・回転状態が吸収に影響を与える可能性がある.
以上2つの研究のように,水素分子の振動・回転状態はプラズマ中及びプラズマ周辺における反応についての大きな要素になり得る.

そこで本研究では,タングステンの熱陰極を用いた水素プラズマにおけるFulcher-α帯の発光スペクトルを計測した.
またそのスペクトルを解析することで,水素分子基底準位の振動・回転状態占有率分布を求めた.
