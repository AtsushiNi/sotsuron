\chapter{緒言}
新たな発電方法として,核融合発電が注目されている.
核融合発電とは重水素と三重水素の核融合反応によって生じるエネルギーを発電に利用する方法である.
核融合発電のメリットとして,海水から燃料を作り出せることや,二酸化炭素を排出しないこと,暴走のリスクが低く高レベルの核廃棄物を出さないことなどが挙げられる\cite{monbu-kagakusyou}.
しかし,実現までの技術的課題は多く,そのうちの一つにプラズマの閉じ込め性能の問題がある.
閉じ込め領域内の水素イオンは,中性粒子となってプラズマの周辺領域へと拡散してしまう.
中性粒子が炉壁やプラズマの周辺機器であるダイバータに衝突すると,そこから水素原子分子が放出される.
水素原子分子はプラズマの中心領域へと戻り,電離して水素イオンになる.
これらの過程は水素リサイクリングと呼ばれ,閉じ込め性能に大きく影響する\cite{hiramatsu}.
しかしこの過程は多くの素過程を含み未だ解明されていない部分も多い.
その中でも分子活性化再結合は,ダイバータの損耗に影響があるとされる重要な素過程である.
そして,分子活性化再結合の反応係数は水素分子の振動・回転状態に大きく依存するため,水素分子がどのような振動・回転状態で放出されるかを明らかにする必要がある\cite{ishihara}.

本研究では,タングステンの熱陰極を用いた水素プラズマにおけるFulcher-α帯の発光スペクトルを計測した.
またそのスペクトルを解析することで,水素分子基底準位の振動・回転状態占有率を求めた.