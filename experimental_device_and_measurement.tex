\chapter{実験装置と発光スペクトルの計測}
\section{プラズマチャンバ}
本研究で用いたプラズマチャンバの外観を図\ref{fig:chamber-picture}に,チャンバの構造を簡略化した図を図\ref{fig:chamber-simple}に示す.
プラズマチャンバは主にICF-152規格のスレンレス製六方クロス管が用いられている.
流量調節バルブは水素ボンベに繋がっており,このバルブでチャンバ内の水素の圧力を調節することができる.
圧力計(PFEIFFER VACUUM PKR251)が取り付けられており,チャンバ内の圧力を調べることができる.
排気口はロータリーポンプ(アネスト岩田,ISP-250B-SV)とターボ分子ポンプ(大阪真空機器製作所,TH162CA)に繋げた.
カソードには直径0.5 mmのタングステンフィラメントを使用し,アノードとして直径50 mmのステンレス平板電極を使用した.
カソードの両端を直流電源(菊水電子工業,REGULATED DC POWER SUPPLY)に繋ぎ電流を流すことで,フィラメントから熱電子を放出させた.
アノードとカソードは別の直流電源(高砂製作所, GP 0110-3)に繋げた.
チャンバには石英窓が取り付けられており,ここにコリメータを付けた光ファイバーを繋ぐことで,プラズマの光を光学系へ取り込んだ.

\section{分光器}
分光器の簡略図を図\ref{fig:spectrometer-picture}に示す.
プラズマからの光は,コリメータを用いて集光したものを光ファイバーでツェルニ・ターナー型分光器(堀場製作所,HR640;焦点距離 640 mm)へ入射させている.
分光器入り口にはスリットを設置し,その幅は50 µmである.
スリットを通過した光はコリメートミラーで平行光となり,回折格子に入射する.
この回折格子は有効幅が10 cm, 刻線数は2400 本/mmである.
回折格子で反射した光は,ミラーで反射した後CCDカメラ(Finger Lakes Instrumentation,ML1109;画素数 2048×506, 画素ピッチ 12 µm, 16 bit)で検出される.
CCDカメラが一度に計測できる波長範囲は波長600 nmで約8.8 nmであり,回折格子をステッピングモーターで回転させることでその範囲を超える波長範囲の計測を行う.
中心波長を602-622 nmの範囲で4 nmおきに変化させ,露光時間150 sでCCDカメラの画像を計測した.

\section{実験の手順}
以下の手順に沿って発光スペクトルを計測した.
\begin{enumerate}
    \item 流量調節バルブを閉じ切って圧力が$10^{-6}$ Torr程度に下がっていることを確認した
    \item 流量調節バルブを少しずつ開き,圧力が$10^{-3}$ Torr程度になるまで水素を注入した.
    \item カソードに20 A程度の電流を流した.
    \item アノード,カソード間に電圧を印加し,プラズマを生成した.アノード,カソード間に流れる電流が2.5 Aになるように調節した.
    \item CCDカメラでスペクトルの撮影を行った.ステッピングモーターを回して撮影する波長範囲を変えながら6 枚の画像を得た.
    \item アノード,カソード間の電源を切ってプラズマを消した後,同様に撮影を行いバックグラウンドのスペクトルを得た.
\end{enumerate}

\section{実験データ}
図\ref{fig:picture-example}に,上述の手順5で撮影した画像の例を示す.
図\ref{fig:spectrum-example}は図\ref{fig:picture-example}の画像を縦方向に0-506 ピクセルの範囲でビニングして得られたスペクトルである.
また,手順6で得たデータのうち,図\ref{fig:spectrum-example}と波長範囲が同じものを図\ref{fig:back-spectrum-example}に示す.
横方向に1027番目のピクセルは不良ピクセルである.
プラズマをつけて得たデータからプラズマを消した状態で得たバックグラウンドのデータを減算することで,このようなノイズを打ち消すことができる.
図\ref{fig:spectrum-example}からバックグラウンドのデータを除いたものが図\ref{fig:true-spectrum-example}である.
以降の解析では,全ての画像に対してこの処理を施したものを使用した.
なお,本研究ではCCDカメラの感度は波長に対して一定であると仮定した.

\section{波長校正}
横方向のピクセル値を波長に変換するために,波長校正を行った.
以降,($J', J''$) = (1, 1), (2, 2), ...の発光線を順にQ1, Q2, ...とする.
まず,($v'-v''$) = (0 - 0)ではQ1-Q7,($v'-v''$) = (1 - 1), (2 - 2)ではQ1-Q5の発光線
に対して光の強度が最大となるピクセル値求めた.
これらの中心ピクセル値とデータベース\cite{H2-spectrum-data}から取得したFulcher-α帯発光線の波長をプロットした(図\ref{fig:pixel-to-wavelength}).
さらに,このデータを二次関数でフィッティングすることで,ピクセル値と波長の対応関係を得た.
波長校正後のスペクトルを図\ref{fig:all-spectrum}に示す.
