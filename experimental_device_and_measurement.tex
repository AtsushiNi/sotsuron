\chapter{実験装置と発光スペクトルの計測}
\section{プラズマチャンバ}
本研究で用いたプラズマチャンバの外観を図\ref{fig:chamber-picture}に,チャンバの構造を簡単にした図を図\ref{fig:chamber-simple}に示す.
プラズマチャンバは主にICF-152規格のスレンレス製六方クロス管が用いられている.
流量調節バルブは水素ボンベに繋がっており,このバルブでチャンバ内の水素の圧力を調節することができる.
圧力計(PFEIFFER VACUUM PKR251)が取り付けられており,チャンバ内の圧力を調べることができる.
排気口はロータリーポンプ(アネスト岩田,ISP-250B-SV)とターボ分子ポンプ(大阪真空機器製作所,TH162CA)に繋げた.
カソードには直径0.5 mmのタングステンフィラメントを使用し,アノードとして直径50 mmの平板電極を使用した.
カソードの両端は直流電源(菊水電子工業,REGULATED DC POWER SUPPLY)に繋ぎ,電流を流すことで,フィラメントから熱電子を放出させた.
アノードとカソードは別の直流電源(高砂製作所, GP 0110-3)に繋げた.
チャンバには石英窓が取り付けられており,ここにコリメータを付けた光ファイバーを繋ぐことで,プラズマの光を光学系へ取り込んだ.

\section{分光器}
分光器の概略図を図\ref{fig:spectrometer-picture}に示す.
プラズマからの光は,コリメータを用いて集光したものを光ファイバーでツェルニ・ターナー型分光器(堀場製作所,HR640)へ入射させている.
分光器入り口にはスリットを設置していて,その幅は50 µmである.
スリットを通過した光はコリメートミラーで平行光となり,回折格子に入射する.
この回折格子は有効幅が10 cm, 刻線数は2400 本/mmである.
回折格子で反射した光は,ミラーで反射した後CCDカメラ(Finger Lakes Instrumentation,ML1109)で検出される.
CCDカメラの仕様\cite{CCD-spec}を表\ref{table:CCD-spec}に示す.
CCDカメラが一度に計測できる波長範囲は8.8 nmだが,回折格子からの光はこの幅を超えて広がるので,回折格子をステッピングモーターで回転させることで,広い波長範囲の計測を行えるようになっている.
また,ステッピングモーターおよびCCDカメラをPythonのプログラムで制御することで,複数の波長範囲に対応する画像を自動的に撮影することができる.

\section{実験の手順}
以下の手順に沿って発光スペクトルを計測した.
\begin{enumerate}
    \item 流量調節バルブを閉じ切って圧力が$10^{-6}$ Torr程度に下がっていることを確認した
    \item 流量調節バルブを少しずつ開き,圧力が$10^{-3}$ Torr程度になるまで水素を注入した.
    \item カソードに電圧を印加し,電流が20 A程度になるように調節した.
    \item アノード,カソード間に電圧を印加し,プラズマを生成した.プラズマ電流が2.5 Aになるように調節した.
    \item CCDカメラでスペクトルの撮影を行った.ステッピングモーターを回して撮影する波長帯を変えながら複数の画像を得た.
    \item アノード,カソード間の電源を切ってプラズマを消した後,同様に撮影を行った.
\end{enumerate}

\section{実験データ}
図\ref{fig:picture-example}に,上述の手順5で撮影した画像の例を示す.
図\ref{fig:spectrum-example}は図\ref{fig:picture-example}の各画素のカウント数を縦に合計したものをプロットしたグラフである.
また,手順6で得たデータのうち,図\ref{fig:spectrum-example}と波長範囲が同じものを図\ref{fig:back-spectrum-example}に示す.
横方向のピクセルが1027のデータが不自然なことが見て分かるが,これはCCDカメラのある一つのピクセルが,光の強度に関わらず異常な値を出力していることが原因である.
プラズマをつけて得たデータからプラズマを消した状態で得たバックグラウンドのデータを減算することで,このようなノイズを打ち消すことができる.
図\ref{fig:spectrum-example}からバックグラウンドのデータを除いたものが図\ref{fig:true-spectrum-example}である.
以降の解析では,全ての画像に対してこの処理を施したものを使用した.

\section{波長校正}
横方向のピクセルを波長に変換するために,波長校正を行った.
まず,光の強度が最大となるピクセルを発光線の中心とし,その中心に対してデータベース\cite{H2-spectrum-data}から取得したFulcher-α帯発光線の波長をプロットした.
さらに,二次関数でフィッティングすることで,ピクセルと波長の対応関係を得た(図\ref{fig:pixel-to-wavelength}).
得られた対応関係からピクセルを波長に変換した.
波長校正後のスペクトルを図\ref{fig:all-spectrum}に示す.
なお,($J', J''$) = (1, 1), (2, 2), ...の発光線を順にQ1, Q2, ...としている.