\chapter{解析結果}

\section{発光強度}
発光強度$I^{dv'N'}_{av''N''}$は,スペクトルの面積を計算して求められる.
得られたスペクトルを歪正規分布関数でフィッティングすることで面積を求めた.
フィッティングの例を図\ref{fig:voigt-fitting-1}に示す.
図中の網掛け部分の面積を使用した.
また,複数の発光線が重なった形をしているものは,歪正規分布関数を足し合わせたものを使いフィッティングすることで分離した.
例として,Q7(0-0)のフィッティング結果を図\ref{fig:voigt-fitting-2}に示す.
はっきりと発光線が区別できるものは,($v'-v''$) = (0 - 0)ではQ7,($v'-v''$) = (1 - 1), (2 - 2)ではQ5までだったので,それら計17本の発光線を解析の対象とした.

\section{上準位ボルツマンプロットおよび回転温度}
式2.8に従って,発光上準位のボルツマンプロットを作成した.
図\ref{fig:upper-boltzmann-plot}は,横軸に$E^{dv'}_{\rm rot}$,縦軸に$\frac{n_{dv'N'}}{\left( 2N'+1 \right) g^{N'}_{\rm as}}$を取ったものである.
縦軸は対数にとり,ボルツマンプロットとした.
$v'=1,2$のグラフの形はほぼ直線であり,発光上準位の占有数がボルツマン分布に従うことがわかる.
一方,$v'=0$のグラフは回転量子数$N'$が大きい部分では形が直線から外れており,ボルツマン分布に従っていないことがわかる.
式2.8を用いてフィッティングすることによって,回転温度$T^{dv'}_{\rm rot}$を求めた.
図\ref{fig:upper-fitting-0},図\ref{fig:upper-fitting-1},図\ref{fig:upper-fitting-2}にフィッティング結果,表\ref{table:upper-temperatures}に得られた回転温度を示す.
ただし,$v'=0$のデータのフィッティングには,グラフが直線とみなせる$N'=1〜5$のみを用いた.

\section{基底準位占有数}
式2.11より電子衝突励起係数$R^{dv'N'}_{XvN}$を求めた.
さらに,式2.16より基底準位の回転温度$T^{Xv}_{\rm rot}$を求めた.
そして式2.15に従いフィッティングを実行し,基底準位の振動温度$T^{X}_{\rm vib}$を求めた.
フィッティング結果を図\ref{fig:fitting-result}に示す.
図の横軸は解析に使用したデータの順番を表し,1〜17まで順に($v',N'$)=(0,1)〜(0.7),(1,1)〜(1,5),(2,1)〜(2,5)となっている.
また,振動温度と回転温度を表\ref{table:fitting-result}に示す.
求まった振動温度と回転温度を式2.14に代入して,基底準位の占有数分布を求めた.
これを図\ref{fig:ground-state-n}に示す.