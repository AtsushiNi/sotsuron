\chapter{解析結果}

\section{発光強度}
発光強度は,発光線スペクトルの面積から求めた.
波長校正に用いた18本の発光線スペクトルを歪正規分布関数でフィッティングし面積を求めた.
例として,Q2(0-0)発光線のフィッティング結果を図\ref{fig:voigt-fitting-1}に示す.
図中の網掛け部分の面積を発光強度とした.
また,複数の発光線が隣接し裾が重畳しているものは,複数の歪正規分布関数を使いフィッティングすることで分離した.
例として,Q7(0-0)発光線のフィッティング結果を図\ref{fig:voigt-fitting-2}に示す.

\section{発光上準位における振動・回転状態占有率}
式 (2.5)を用いて,発光上準位における振動・回転状態占有率を求めた.
そしてその占有率を回転と核スピンの縮退度で割ったものを,回転エネルギーに対してプロットした.
それを図\ref{fig:upper-boltzmann-plot}に示す.
なお,縦軸は対数にとり,ボルツマンプロットとした.

図\ref{fig:upper-boltzmann-plot}より,$v'=0$のグラフは$J'=5$付近で傾きが変化しており,1温度のボルツマン分布では振動・回転状態占有率を表せないことが分かった.
そこで,発光上準位の振動・回転状態占有率が2つの回転温度のボルツマン分布の和で近似できるとして次式でフィッティングを行った.
\begin{equation}
    \frac{n_{dv'J'}}{\left( 2J'+1 \right)g^{J'}_{\rm as}} = C'' \left[ (1-a^{v'})\exp \left( - \frac{E^{dv'}_{\rm rot}(J')}{k_{\rm B} T^{dv'}_{\rm rot,1}} \right)+ a^{v'} \exp \left( - \frac{E^{dv'}_{\rm rot}(J')}{k_{\rm B} T^{dv'}_{\rm rot,2}} \right) \right]
\end{equation}
ここで,$C''$は$v'$, $J'$に依存しない定数,$T^{dv'}_{\rm rot,1}$は低回転温度,$T^{dv'}_{\rm rot,2}$は高回転温度,$a^{v'}$は高回転温度を持つ振動・回転状態占有数の比率である.
この式を用いて$v'=0$のグラフをフィッティングした結果を,図\ref{fig:two-boltzmann-fitting}に示す.
また,この時のフィッティングパラメータを表\ref{table:two-boltzmann-parameters}に示す.

\section{基底準位における振動・回転状態占有率}
$J \leq 5$では基底準位の振動・回転状態占有率がボルツマン分布に従うと仮定し,コロナモデルを用いて基底準位の振動・回転状態占有率を求めた.
まず,式 (2.13)より電子衝突励起係数を求めた.
なお,この時に式 (2.14)で用いた電子温度はラングミュアプローブを用いた実験で得た値である7 eV\cite{yun}を用いた.
そして,式 (2.17)の右辺をフィッティング関数,$T^{Xv}_{\rm rot}$, $T^X_{\rm vib}$をフィッティングパラメータとして,4.2章で求めた発光上準位における$J' \leq 5$での振動・回転状態占有率のデータをフィッティングした.
このフィッティング結果を図\ref{fig:fitting-result}に示す.
図の横軸は解析に使用したデータの順番を表し,1〜15まで順に($v',J'$)=(0,1)〜(0.5),(1,1)〜(1,5),(2,1)〜(2,5)となっている.
また,この時のフィッティングパラメータの値を表\ref{table:fitting-result}に示す.
求まった振動温度と回転温度を式 (2.16)に代入して,基底準位の振動・回転状態占有率を求め,その回転エネルギーに対する依存性を図\ref{fig:ground-state-n}に示す.
ただし,振動準位$v=2$の振動・回転状態占有率については,回転温度の誤差が大きく精確に求めることができなかったため,記載していない.