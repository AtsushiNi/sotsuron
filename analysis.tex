\chapter{解析結果}

\section{発光強度}
発光強度は,発光線スペクトルの面積を計算して求められる.
得られた発光線スペクトルを歪正規分布関数でフィッティングすることで面積を求めた.
例として,Q2(0-0)のフィッティング結果を図\ref{fig:voigt-fitting-1}に示す.
図中の網掛け部分の面積が発光強度である.
また,複数の発光線が重なった形をしているものは,歪正規分布関数を足し合わせたものを使いフィッティングすることで分離した.
例として,Q7(0-0)のフィッティング結果を図\ref{fig:voigt-fitting-2}に示す.
はっきりと発光線が区別できるものは,($v'-v''$) = (0 - 0)ではQ7,($v'-v''$) = (1 - 1), (2 - 2)ではQ5までだったので,それら計17本の発光線を解析の対象とした.

\section{発光上準位における振動・回転状態占有率のボルツマンプロットおよび回転温度}
式2.5を用いて,上準位における振動・回転状態占有率を求めた.
そしてその占有率を,横軸に$E^{dv'}_{\rm rot}$,縦軸に$\frac{n_{dv'J'}}{\left( 2J'+1 \right) g^{J'}_{\rm as}}$を取ったグラフにプロットした.
それを図\ref{fig:upper-boltzmann-plot}に示す.
なお,縦軸は対数にとり,ボルツマンプロットとした.
$v'=1,2$のグラフの形はほぼ直線であり,発光上準位の占有数がボルツマン分布に従うことがわかる.
一方,$v'=0$のグラフは回転量子数が大きい部分では形が直線から外れており,ボルツマン分布に従っていないことがわかる.
式2.9を用いてフィッティングすることによって,回転温度を求めた.
ただし,$v'=0$のデータのフィッティングには,グラフが直線とみなせる$J'=1〜5$のみを用いた.
図\ref{fig:upper-fitting-0},図\ref{fig:upper-fitting-1},図\ref{fig:upper-fitting-2}にフィッティング結果,表\ref{table:upper-temperatures}に得られた回転温度を示す.

\section{基底準位における振動・回転状態占有率}
式2.13より電子衝突励起係数を求めた.
さらに,式2.18より基底準位の回転温度を求め,これらの値を式2.17に代入した.
式2.17の左辺は式2.5によって計算できるが右辺の値は$T^X_{vib}$によって変化するので,左辺の値と右辺の値が一番近くなるように$T^X_{vib}$を決定した.
このフィッティング結果を図\ref{fig:fitting-result}に示す.
図の横軸は解析に使用したデータの順番を表し,1〜17まで順に($v',J'$)=(0,1)〜(0.7),(1,1)〜(1,5),(2,1)〜(2,5)となっている.
また,振動温度と回転温度を表\ref{table:fitting-result}に示す.
求まった振動温度と回転温度を式2.16に代入して,基底準位の占有率を求めた.
これを図\ref{fig:ground-state-n}に示す.