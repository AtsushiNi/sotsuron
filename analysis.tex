\chapter{解析結果}

\section{発光強度}
発光強度は,発光線スペクトルの面積から求めた.
波長校正に用いた17本の発光線スペクトルを歪正規分布関数でフィッティングし面積を求めた.
例として,Q2(0-0)のフィッティング結果を図\ref{fig:voigt-fitting-1}に示す.
図中の網掛け部分の面積を発光強度とした.
また,複数の発光線が隣接し裾が重畳しているものは,複数の歪正規分布関数を使いフィッティングすることで分離した.
例として,Q7(0-0)のフィッティング結果を図\ref{fig:voigt-fitting-2}に示す.

\section{発光上準位における振動・回転状態占有率および回転温度}
式 (2.5)を用いて,発光上準位における振動・回転状態占有率を求めた.
そしてその占有率を回転と核スピンの縮退度で割ったものを,回転エネルギーに対してプロットした.
それを図\ref{fig:upper-boltzmann-plot}に示す.
なお,縦軸は対数にとし,ボルツマンプロットとした.
図\ref{fig:upper-boltzmann-plot}を直線でフィッティングし,直線の傾きを得ることによって,式 (2.9)から回転温度を求めた.
ただし,$v'=0$のデータのフィッティングには,グラフが直線でよく近似できる$J'=1〜5$のデータのみを用いた.
図\ref{fig:upper-fitting}にフィッティング結果,表\ref{table:upper-temperatures}に得られた回転温度を示す.

\section{基底準位における振動・回転状態占有率}
式 (2.13)より電子衝突励起係数を求めた.
なお,この時に用いた電子温度はラングミュアプローブを用いた実験で得た値である7 eV\cite{yun}を用いた.
さらに,式 (2.18)より基底準位の回転温度を求め,これらの値を式 (2.17)に代入した.
4.2章で求めた発光上準位の振動・回転状態占有率のデータに対し,式 (2.17)を$T^X_{\rm vib}$をパラメータとしてフィッtλイングし$T^X_{\rm vib}$を決定した.
このフィッティング結果を図\ref{fig:fitting-result}に示す.
図の横軸は解析に使用したデータの順番を表し,1〜17まで順に($v',J'$)=(0,1)〜(0.7),(1,1)〜(1,5),(2,1)〜(2,5)となっている.
また,発光上準位の振動温度と回転温度を表\ref{table:fitting-result}に示す.
求まった振動温度と回転温度を式 (2.16)に代入して,基底準位の占有率を求めた.
基底準位における占有率の,回転エネルギーに対する依存性を図\ref{fig:ground-state-n}に示す.