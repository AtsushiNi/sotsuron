\begin{abstract}
タングステンのフィラメントを熱陰極とする,グロー放電プラズマの波長600-630 nmの水素分子発光スペクトルを,ツェルニ・ターナー型分光器とCCDカメラを用いて計測した.
得られたスペクトルから,Fulcher-α帯Q枝の発光線を特定し,その発光強度を求めた.
発光強度から発光上準位の振動・回転状態占有率を計算し,回転エネルギーに対するボルツマンプロットを作成した.
振動準位$v'=0$の振動・回転状態占有率は2温度のボルツマン分布に従うことが分かった.
コロナモデルを適用するとともに,基底準位の低回転量子数域において振動・回転状態占有率にボルツマン分布を仮定することで,基底準位の振動温度と回転温度を求めた.
そして,基底準位の振動・回転状態占有率を求め,回転エネルギーに対するボルツマンプロットを示した.
また,本研究対象のプラズマとは種類の異なるプラズマであるLHDで行われた計測結果との比較を行った.
発光上準位の振動準位$v'=0$では,低回転温度はLHDと同程度である一方,高回転温度はLHDの約1.4 倍であった.
基底準位の低回転量子数域では,回転温度はLHDと同程度であった.
さらに,振動温度が回転温度に比べ数倍高いという点でもLHDと共通していることが分かった.
\end{abstract}