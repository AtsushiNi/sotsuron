\begin{abstract}
タングステンのフィラメントを熱陰極とする,グロー放電プラズマの波長600-630 nmの水素分子発光スペクトルを,ツェルニ・ターナー型分光器とCCDカメラを用いて計測した.
得られたスペクトルから,Fulcher-α帯Q枝の発光線を特定し,その発光強度を求めた.
発光強度から発光上準位の振動・回転状態占有率を計算し,回転エネルギーに対するボルツマンプロットを作成した.
発光上準位の占有率にボルツマン分布を仮定しボルツマンプロットをフィッティングすることによって,発光上準位の回転温度を求めた.
コロナモデルを適用するとともに基底準位の振動・回転状態占有率にボルツマン分布を仮定することで,基底準位の振動温度と回転温度を求めた.
そして,基底準位の振動・回転状態占有率を求め,回転エネルギーに対するボルツマンプロットを示した.
また,大型ヘリカル装置(LHD)で計測されたデータとの比較を行うことにより,プラズマの種類は異なるが,基底準位の回転温度より振動温度の方がはるかに高いことと,振動準位が大きくなるにつれて回転温度が低くなることの2点は共通しているということが分かった.
\end{abstract}