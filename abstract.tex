\begin{abstract}
タングステンのフィラメントを熱陰極とする,グロー放電プラズマの水素分子発光スペクトルを,ツェルニ・ターナー型分光器とCCDカメラを用いて計測した.
得られたスペクトルから,Fulcher-α帯Q枝の発光線を特定し,その発光強度を計算した.
発光強度から発光上準位の振動・回転状態占有率を計算し,回転エネルギーに対するボルツマンプロットを作成した.
発光上準位の占有率にボルツマン分布を仮定しボルツマンプロットを直線でフィッティングすることによって,その傾きから発光上準位の回転温度を求めた.
コロナモデルを適用するとともに基底準位の振動・回転状態占有率にもボルツマン分布を仮定することで,基底準位の振動温度を求めた.
最後に,基底準位の振動・回転状態占有率を求め,回転エネルギーに対する依存性を示した.
また,発光上準位における占有率のボルツマンプロットが直線から一部外れていることから,水素分子の回転温度が視線方向に一定ではないことが考えられた.
\end{abstract}