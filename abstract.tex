\begin{abstract}
タングステンのフィラメントを熱陰極とする,グロー放電プラズマの波長600-630 nmの水素分子発光スペクトルを,ツェルニ・ターナー型分光器とCCDカメラを用いて計測した.
得られたスペクトルから,Fulcher-α帯Q枝の発光線を特定し,その発光強度を求めた.
発光強度から発光上準位の振動・回転状態占有率を計算し,回転エネルギーに対するボルツマンプロットを作成した.
発光上準位の占有率にボルツマン分布を仮定しボルツマンプロットをフィッティングすることによって,発光上準位の回転温度を求めた.
コロナモデルを適用するとともに基底準位の振動・回転状態占有率にボルツマン分布を仮定することで,基底準位の振動温度と回転温度を求めた.
そして,基底準位の振動・回転状態占有率を求め,回転エネルギーに対するボルツマンプロットを示した.
また,本研究対象のプラズマとは種類の異なるプラズマであるLHDで行われた計測結果との比較を行った.
基底準位の回転温度に比べ基底準位の振動温度が,本研究対象のプラズマでは7.2 倍,LHDでは20.3 倍と共にはるかに高いことが分かった.
さらに,基底準位において振動準位が大きくなるにつれて回転温度が低くなる点でも2つのプラズマは共通しているということが分かった.
\end{abstract}