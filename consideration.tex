\chapter{考察}
発光上準位及び基底準位における振動・回転状態について,LHDでのデータとの比較を行う.
なお,LHDと本研究で用いたプラズマの違いを表\ref{table:LHD-and-this-plasma}にまとめた\cite{ishihara}.

\section{発光上準位に関するLHDでのデータとの比較}
図\ref{fig:two-boltzmann-compare}に本研究対象のプラズマとLHDでの発光上準位$v'=0$における振動・回転状態占有率を示す.
図より,振動準位$v'=0$ではLHDと同様の2温度分布が確認できる.
また,表\ref{table:two-boltzmann-compare}に本研究対象のプラズマとLHDでの,発光上準位$v'=0$における低回転温度,高回転温度,及び高回転温度を持つ振動・回転状態占有数の比率を示す.
表より,低回転温度はLHDとの差が小さい一方,高回転温度はLHDの約1.4倍高いことがわかる.
また,本研究対象のプラズマでは高回転温度を持つ振動・回転状態占有数の割合は,LHDの1割未満であった.

\section{基底準位に関するLHDでのデータとの比較}
図\ref{fig:ground-compare}に本研究対象のプラズマとLHDでの基底準位における振動・回転状態占有率を示す.
また,表\ref{table:ground-result-compare}は,本研究で求めた基底準位の振動・回転温度と,LHDでの基底準位の振動・回転温度\cite{ishihara}である.
ただしLHDでの解析は2温度ボルツマン分布を仮定しているが,低回転温度のみを記載している.
この表から,$v=2$の回転温度は誤差が大きくLHDとの比較は難しいが,$v=0,1$の回転温度はLHDと同程度であることがわかる.
さらに,振動温度は回転温度に比べて本研究対象のプラズマでは4.3 倍,LHDでは3.2 倍と共にはるかに高いということが分かった.
