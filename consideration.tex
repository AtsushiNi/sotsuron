\chapter{考察}
発光上準位及び基底準位における振動・回転状態について,LHDでのデータとの比較を行う.
なお,LHDと本研究で用いたプラズマの違いを表\ref{table:LHD-and-this-plasma}にまとめた\cite{ishihara}.

\section{発光上準位に関するLHDでのデータとの比較}
発光上準位の振動・回転状態占有率に関して,LHDでのデータと比較する.
図\ref{fig:ishihara-upper-boltzmann}はLHDにおける発光上準位の振動・回転状態占有率
の回転エネルギー依存性を示している\cite{ishihara}.
図中の点線は,以下の式で表される2温度のボルツマン分布の和によるフィッティング結果である.
\begin{equation}
    \frac{n_{dv'J'}}{\left( 2J'+1 \right)g^{J'}_{\rm as}} = (1-a^{v'})\exp \left( - \frac{E^{dv'}_{\rm rot}(J')}{k_{\rm B} T^{dv'}_{\rm rot,1}} \right)+ a^{v'} \exp \left( - \frac{E^{dv'}_{\rm rot}(J')}{k_{\rm B} T^{dv'}_{\rm rot,2}} \right)
\end{equation}
ここで,$T^{dv'}_{\rm rot,1}$は低回転温度,$T^{dv'}_{\rm rot,2}$は高回転温度,$a^{v'}$は高回転温度を持つ振動・回転状態占有率の比率である.
高回転量子数帯を含めた占有数がよくフィッティングできていることが分かる.

$v'=1,2$の振動準位に関して,今回の実験では$J' \leq 5$という低回転温度を持つ回転量子数帯のデータしか取れなかったため,発光上準位の占有率が1温度のボルツマン分布に従うとしたが,より高い分解能の分光器を用いて高回転量子数帯までデータを取ると,発光上準位の振動・
回転状態占有率はLHDと同様に2温度のボルツマン分布に従うと予想できる.

$v'=0$の振動準位に関して,4.2章では図\ref{fig:upper-fitting}において直線でよく近似できる$J'=1$-5のみを用いたが,ここでは$J'=6,7$のデータも含めてLHDでの解析と同様に2温度のボルツマン分布の和でフィッティングを行った.フィッティング結果を図\ref{fig:upper-two-fitting}に示す.
また,このフィッティングにより求まった$T^{dv'}_{\rm rot,1}$, $T^{dv'}_{\rm rot,2}$, $a^{v'}$と,LHDでの$T^{dv'}_{\rm rot,1}$, $T^{dv'}_{\rm rot,2}$, $a^{v'}$を表\ref{table:two-result-table}に示す.
図\ref{fig:upper-two-fitting}から,回転エネルギーがLHDと同じ0.1 eV付近でフィッティング関数の傾きが小さくなっていることが分かった.

\section{基底準位に関するLHDでのデータとの比較}
表\ref{table:ground-result-compare}は,本研究で求めた基底準位の振動・回転温度と,LHDでの基底準位の振動・回転温度\cite{ishihara}である.
ただしLHDでの解析は2温度ボルツマン分布を仮定しているが,低回転量子数帯に関係している低温度を記載している.
この表から,本研究対象のプラズマとLHDではプラズマの種類が違っても,回転温度に比べて振動温度が本研究対象のプラズマでは7.2 倍,LHDでは20.3 倍と共にはるかに高いということが分かった.
さらに振動量子数が大きくなるにつれて回転温度が低くなるという点でも2つのプラズマは共通するということが分かった.