\chapter{考察}
図\ref{fig:upper-fitting-0}に関して,発光上準位の振動・回転状態占有率がボルツマン分布に従うとするとこのグラフは直線になるはずだが,実際は$v'=0$の$J'=6,7$の点は大きな値になっている.
このことから,回転温度は視線方向に一定ではないことが考えられる.
高回転量子数域を含めた発光上準位の占有数のボルツマンプロットが直線から外れることは,他のプラズマでも報告されている.
図\ref{fig:ishihara-upper-boltzmann}は,岐阜県土岐市の核融合科学研究所にある大型ヘリカル装置(LHD)でのプラズマにおける,発光上準位における占有数のボルツマンプロットである\cite{ishihara}.
図中の点線は高低2温度のボルツマン分布を仮定したフィッティング結果であり,良く近似できていることが分かる.
また,発光上準位の占有数分布が2温度のボルツマン分布で良く近似できることは,放電の種類に関わらず複数計測されている\cite{ishihara, two-temperature-1, two-temperature-2}.
本研究の対称としたプラズマでも,より光分解能の分光器を用いて高回転量子数域までデータを取り,2温度のボルツマン分布を仮定することで,正確な占有率を計算することができると考えられる.