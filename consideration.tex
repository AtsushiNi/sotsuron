\chapter{考察}
発光上準位及び基底準位における振動・回転状態について,LHDでのデータとの比較を行う.
なお,LHDと本研究で用いたプラズマの違いを表\ref{table:LHD-and-this-plasma}にまとめた\cite{ishihara}.

\section{発光上準位に関するLHDでのデータとの比較}
図\ref{fig:upper-fitting-0}に関して,発光上準位の振動・回転状態占有率が完全に1温度のボルツマン分布に従うならばこのグラフは直線状に並ぶはずだが,実際は$J'=6,7$の点は大きな値になっている.
このことについてLHDでのデータと比較する.
図\ref{fig:ishihara-upper-boltzmann}はLHDにおける発光上準位の振動・回転状態占有数の回転エネルギー依存性を示している\cite{ishihara}.
回転エネルギーが本研究と同じ0.1 eV前後でフィッティング関数の傾きが大きくなっていることが読み取る.
また,高回転量子数帯を含めた占有数が2温度のボルツマン分布の和によってよくフィッティングできていることが分かる.
今回の実験では$J' \leq 7$という低回転量子数帯のデータしか取れなかったため,発光上準位の占有率が1温度のボルツマン分布に従うとしたが,より光分解能の分光器を用いて高回転量子数帯までデータを取ると,LHDと同様に2温度のボルツマン分布に従うと予想できる.

\section{基底準位に関するLHDでのデータとの比較}
表\ref{table:ground-result-compare}は,本研究で求めた基底準位の振動・回転温度と,LHDでの基底準位の振動・回転温度\cite{ishihara}である.
ただしLHDでの解析は2温度ボルツマン分布を仮定しているが,低回転量子数帯に関係している低い方の温度を記載している.
この表から,本研究対象のプラズマとLHDではプラズマの種類が違っても,回転温度に比べて振動温度がはるかに高いということと,振動量子数が大きくなるにつれて回転温度が高くなるという2点は共通するということが分かった.

\begin{comment}
図\ref{fig:upper-fitting-0}に関して,発光上準位の振動・回転状態占有率が完全にボルツマン分布に従うとするとこのグラフは直線になるはずだが,実際は$v'=0$の$J'=6,7$の点は大きな値になっている.
このことから,回転温度は視線方向に一定ではないことが考えられる.
高回転量子数域を含めた発光上準位の占有数のボルツマンプロットが直線から外れることは,他のプラズマでも報告されている.
図\ref{fig:ishihara-upper-boltzmann}は,岐阜県土岐市の核融合科学研究所にある大型ヘリカル装置(LHD)でのプラズマにおける,発光上準位における占有数のボルツマンプロットである\cite{ishihara}.
図中の点線は高低2温度のボルツマン分布を仮定したフィッティング結果であり,良く近似できていることが分かる.
また,発光上準位の占有数分布が2温度のボルツマン分布で良く近似できることは,放電の種類に関わらず複数計測されている\cite{ishihara, two-temperature-1, two-temperature-2}.
本研究の対称としたプラズマでも,より光分解能の分光器を用いて高回転量子数域までデータを取り,2温度のボルツマン分布を仮定することで,正確な占有率を計算することができると考えられる.
\end{comment}