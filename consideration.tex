\chapter{考察}

\section{発光強度の誤差}
本研究で求めた占有率分布の誤差について,スペクトルから発光強度を出す際に求めた面積の誤差が考えられる.
例えば図\ref{fig:voigt-fitting-1}に示すQ2(0-0)発光線のように,発光線データは波長に対して左右非対称な形をしていた.
この原因を特定し,影響を補正したスペクトルから面積を求めればより正確な発光強度が得られたと思われる.
また,発光線の分離ができなかったデータがあることも面積の誤差に含まれる.
図\ref{fig:voigt-fitting-3}はQ3(0-0)の発光線をフィッティングした図だが,発光線の左側に誤差があることが分かる.
Q3(0-0)の波長は603.1909nmであるが,この左側には5分の1程度の強度を持つP2(2-2)発光線が603.1465nmにあり乗畳している\cite{kyokaisou}.
このような誤差をなくすには,より高い分解能を持つ分光器を用い,発光線を正確に分離することが必要となる.

\section{モデル化の妥当性}
図\ref{fig:fitting-result}のフィッティングのずれの原因には,発光強度の誤差の他に,発光上準位と基底準位のモデル化にもあると思われる.
図\ref{fig:upper-fitting-0}に関して,上準位の占有数がボルツマンプロットに従うとするとこのグラフは直線になるはずだが,実際は$v'=0$の$N'=6,7$の点は大きな値になっている.このグラフを説明する考え方として,発光上準位の占有数は2温度のボルツマン分布に従うというものがある.
これは,二つの異なる温度のボルツマン分布に従う分子が視線上に存在するということである.
図\ref{fig:ishihara-upper-boltzmann}は,岐阜県土岐市の核融合科学研究所にある大型ヘリカル装置(LHD)でのプラズマにおける,発光上準位のボルツマンプロットである\cite{ishihara}.
図中の点線は高低2温度のボルツマン分布のフィッティング結果であり,良く近似できていることが分かる.
また,発光上準位の占有数分布が2温度のボルツマン分布で良く近似できることは,放電の種類に関わらず複数計測されている\cite{ishihara, two-temperature-1, two-temperature-2}.本研究の対称としたプラズマでも,より光分解能の分光器を用いて高回転量子数域までデータを取り,2温度のボルツマン分布を仮定することで,正確な占有数を計算することができると予想できる.