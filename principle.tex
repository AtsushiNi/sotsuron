\chapter{原理}

\section{水素分子}
水素分子の発光分光計測に必要となる基本事項を説明する.
\subsection{分子のエネルギー準位}
Fulcher-α遷移に関わる水素分子のエネルギー準位について説明する.
図\ref{fig:energy-level}に水素分子のポテンシャル曲線を示す.
横軸は分子の核間距離,縦軸はポテンシャルエネルギーである.

図中の記号X, d, aは分子中の電子のエネルギー準位に対応し,基底準位をXと表し,基底準位と同じスピン多重度を持つ電子励起準位はエネルギーの低い順にA,B,C,...,異なるスピン多重度を持つ電子励起準位はa,b,c,...と表す.
また$\Sigma$や$\Pi$といった記号は,軌道角運動量$L$の分子軸方向への射影成分$M_L$の大きさ$\Lambda = |M_L|=$0,1,2,3,...に対応しており,順に$\Sigma,\Pi,\Delta,\Phi,...$を用いる\cite{bunsibunko-no-kiso}.
その左上の数字はスピン多重度$2S+1$である\cite{bunsibunko-no-kiso}.
分子軸を含む平面についての鏡面反転操作による電子波動関数の符号変化の有無に応じて上付き添え字+または-を付けており,分子中心についての反転操作による符号変化の有無に応じて下付き添え字uまたはgを付けている\cite{bunsibunko-no-kiso}.

それぞれの電子励起準位について振動のエネルギー準位が存在する.
図中の$v, v', v''$はそれぞれ,X, d, a準位の振動量子数である.
さらに,それぞれの振動励起準位について回転のエネルギー準位が存在し,図中の$J, J', J''$はそれぞれX, d, a準位の回転量子数である.

プラズマ中では基底準位$X^1 \Sigma^+_g$にある水素分子が電子と非弾性衝突することで発光上準位$d^3 \Pi^-_u$へ励起される.
その励起された分子が発光下準位$a^3 \Sigma^+_g$へ自然放出する際に発光するスペクトル帯が本研究の計測対象とするFulcher-α帯である.

\subsection{分子の振動・回転エネルギー}
原子核は電子に比べて数千倍重いために分子の振動・回転は電子の運動に比べて十分遅くなり,電子が原子核の位置変化に対して瞬時に追随できるとする近似をBorn-Oppenheimer近似という.
この近似を用いると,水素分子を構成する全ての粒子に対してシュレディンガー方程式を同時に解くのではなく,分子の電子波動関数と振動・回転の波動関数を分離することで,振動・回転準位エネルギーを求めることが可能になる.
回転による遠心力ひずみを考慮した回転準位エネルギーは以下のように近似的に表すことができる\cite{bunsibunko-no-kiso}.
\begin{equation}
    E^v_{\rm rot} \cong B^v J (J+1) - D_{\rm e} [J(J+1)]^2
\end{equation}
ここで$B^{\rm v}$は回転定数であり,平衡点における回転定数$B_{\rm e}$と回転定数の1次の補正項$\alpha_{\rm e}$を用いて$B^{\rm v}=B_{\rm e} - \alpha_{\rm e} (v + 1/2)$と近似される.
$D_{\rm e}$は遠心力ひずみ定数である.
また,振動準位エネルギーは以下の近似式で表すことができる.
\begin{equation}
    E_{\rm vib} \cong \overline{\omega}_{\rm e} \left(v + \frac{1}{2} \right) - \overline{\omega}_{\rm e} \chi_{\rm e} \left(v + \frac{1}{2} \right)^2
\end{equation}
ここで$\overline{\omega}_{\rm e}$は振動定数,$\chi_{\rm e}$は2次の補正項である.
表\ref{table:molecular-constants}にまとめたこれらの分子定数はデータベース\cite{nist}から得られる.

\subsection{Fulcher-$\alpha$帯遷移の選択則}
分子の電子準位間の遷移には一般的に振動回転準位間の遷移が伴うが,そこには選択則が存在する\cite{bunsibunko-no-kiso}.
回転量子数は$\Delta J = J' - J'' = -1,0,1$を満たすが,$J'=J''=0$は禁制遷移である.
$\Delta J$に対応するスペクトル枝をそれぞれP枝,Q枝,R枝と呼ぶが,本研究ではQ枝について分析を行うので,$\Delta J = 0$である.

\section{発光上準位及び基底準位の振動・回転状態占有率}
\subsection{発光上準位の振動・回転状態占有率と発光強度の関係}
発光上準位の振動・回転状態占有率$n_{dv'J'}$は発光上準位の振動・回転状態占有数$N_{dv'J'}$を用いて下式で表す.
\begin{equation}
    n_{dv'J'} = \frac{N_{dv'J'}}{\sum_{v'J'} N_{dv'J'}}
\end{equation}
$N_{dv'J'}$は発光強度$I^{dv'J'}_{av''J''}$を用いて
\begin{equation}
    N_{dv'J'}=\frac{I^{dv'J'}_{av''J''}\lambda^{dv'J'}_{av''J''}}{hc}\frac{1}{A^{dv'J'}_{av''J''}}
\end{equation}
と表せるので,
\begin{equation}
    n_{dv'J'} = \frac{1}{\sum_{v'J'} \frac{I^{dv'J'}_{av''J''}\lambda^{dv'J'}_{av''J''}}{A^{dv'J'}_{av''J''}}} \frac{I^{dv'J'}_{av''J''}\lambda^{dv'J'}_{av''J''}}{A^{dv'J'}_{av''J''}}
\end{equation}
と書ける.
ここで$\lambda^{dv'N'}_{av''N''}$は発光線の波長,$h$はプランク定数,$c$は光速である.
$A^{dv'J'}_{av''J''}$は自然発光係数であり
\begin{equation}
    A^{dv'J'}_{av''J''} = \frac{16 \pi^3}{3h \epsilon_0 {\lambda^{dv'J'}_{av''J''}}^3 } \overline{R_{\rm e}}^2 q^{dv'}_{av''} S^{J'}_{J''} \frac{1}{2J'+1}
\end{equation}
と表せる\cite{PRnoijousei}.
$\epsilon_0$は真空の誘電率,$\overline{R_{\rm e}}$は電気双極子モーメント,$q^{dv'}_{av''}$はFranck-Condon因子\cite{franck-condon},$S^{J'}_{J''}$はH\"{o}nl-London因子,$2J'+1$は回転の縮退度である.
Fulcher-α帯ではH\"{o}nl-London因子は
\begin{equation}
    S^{J'}_{J''} = \begin{cases}
        \frac{J'}{2} & (P枝)\\
        \frac{2J'+1}{2} & (Q枝)\\
        \frac{J'+1}{2} & (R枝)
    \end{cases}
\end{equation}
である\cite{PRnoijousei}.

\subsection{発光上準位における振動・回転状態占有率および回転温度}
発光上準位の振動・回転状態占有数がボルツマン分布に従うと仮定すると,その占有率は
\begin{equation}
    n_{dv'J'} = \frac{ \left(  2J'+1 \right) g^{J'}_{\rm as} \exp \left(-\frac{E^{dv'}_{\rm rot}(J')}{k_{\rm B} T^{dv'}_{\rm rot}} \right) }{ \sum_{J'} \left(  2J'+1 \right) g^{J'}_{\rm as} \exp \left(-\frac{E^{dv'}_{\rm rot}(J')}{k_{\rm B} T^{dv'}_{\rm rot}} \right)}
\end{equation}
と表せる.
$(2J'+1)$, $g^{J'}_{\rm as}$, $E^{dv'}_{\rm rot}$, $T^{dv'}_{\rm rot}$は,それぞれ発光上準位の回転の縮退度,核スピンの縮退度,回転エネルギー,回転温度であり,$k_{\rm B}$はボルツマン定数である.
両辺対数をとると,
\begin{equation}
    \ln \left[ \frac{n_{dv'J'}}{\left( 2J'+1 \right) g^{J'}_{\rm as}} \right]
    = const - \frac{E^{dv'}_{\rm rot}(J')}{k_{\rm B} T^{dv'}_{\rm rot}}
\end{equation}
となる.
横軸を$E^{dv'}_{\rm rot}(J')$,縦軸を$\frac{n_{dv'J'}}{\left( 2J'+1 \right) g^{J'}_{\rm as}}$としてプロットし,縦軸を対数に取った図を作成して直線でフィッティングすると,その傾きから$T^{dv'}_{\rm rot}$を求めることができる.

\subsection{コロナモデル}
電子密度が十分に小さい領域では,発光上準位の占有数変化を,電子衝突励起による基底準位からの流入量と自然放出脱励起による発光下準位への流出量で表すことができる.
これをコロナ平衡といい,以下の式で表される\cite{PRnoijousei}.
\begin{equation}
    \frac{\partial N_{dv'J'}}{\partial t} = n_{\rm e} \sum_{v, J} \left[ N_{XvJ} R^{dv'J'}_{XvJ} \right] - N_{dv'J'} \sum_{v''J''} A^{dv'J'}_{av''J''}
\end{equation}
ここで$n_{\rm e}$は電子密度であり,$N_{XvJ}$は基底準位の振動・回転状態占有数である.
右辺第2項の総和記号は選択則($J'-J''=-1,0,1$)を満たす$v'',J''$についての和であることに注意する.
左辺を0とした定常解から,コロナモデルを得られる.
\begin{equation}
    n_{\rm e} \sum_{v, J} \left[ N_{XvJ} R^{dv'J'}_{XvJ} \right] = N_{dv'J'} \sum_{v''J''} A^{dv'J'}_{av''J''}
\end{equation}
式 (2.3)より
\begin{equation}
    n_{\rm e} \left( \sum_{v, J} N_{XvJ} \right) \sum_{v, J} \left[ n_{XvJ} R^{dv'J'}_{XvJ} \right] = n_{dv'J'} \left( \sum_{v', J'} N_{dv'J'} \right) \sum_{v''J''} A^{dv'J'}_{av''J''}
\end{equation}
ここで$n_{XvJ}$は基底準位の振動・回転状態占有率であり,$n_{XvJ} = N_{XvJ} / \sum_{v,J} N_{XvJ}$である。
$R^{dv'J'}_{XvJ}$は電子衝突励起係数であり,Born-Oppenheimer近似を仮定すると
\begin{equation}
    R^{dv'J'}_{XvJ} = q^{dv'}_{Xv} \left< \sigma^{FC}_{v' \leftarrow v} v_{\rm e} \right> a^{1J'}_{0J} \delta^{g^{J'}_{as}}_{g^J_{as}}
\end{equation}
と書ける\cite{PRnoijousei}.
$\sigma^{FC}_{v' \leftarrow v}$は電子衝突励起断面積であり,$v_{\rm e}$はマクスウェル分布を仮定した電子の速度分布である.
$a^{1J'}_{0J}$は回転構造の分岐比,$\delta^{g^{J'}_{as}}_{g^J_{as}}$はクロネッカーのデルタである.
電子衝突断面積について,本研究では次式のようにFranck-Condon原理と電子衝突励起確率のボルツマン則(エネルギー差が小さい遷移ほど励起確率が大きい)を仮定したスケーリングを用いた\cite{kyokaisou}.
\begin{equation}
    \left< \sigma^{FC}_{v' \leftarrow v} v_{\rm e} \right> \propto \left< \sigma^{FC}_{0 \leftarrow 0} v_{\rm e} \right> \times \exp \left( -\frac{(E^d_{\rm vib} (v') - E^d_{\rm vib} (0))-(E^X_{\rm vib} (v) - E^X_{\rm vib} (0))}{k_{\rm B} T_{\rm e}} \right)
\end{equation}
$E^d_{\rm vib} (v'), E^X_{\rm vib} (v)$はそれぞれ発光上準位と基底準位の振動エネルギーで,式 (2.2)を用いて算出した.
$T_{e}$は電子温度である.
回転構造の分岐比は
\begin{equation}
    a^{1J'}_{0J} = \sum_r \overline{Q'_{r}} (2J'+1) \left( \begin{array}{ccc} J' & r & J \\ 1 & -1 & 0 \end{array} \right)^2
\end{equation}
と表せる\cite{kyokaisou}が,ここで$\overline{Q'_{r}}$は電子速度で平均化した部分断面積で,実験的に求められた値\cite{senkusya}を使用した.
なお,右辺の行列はWignerの3j記号である.
クロネッカーのデルタ$\delta^{g^{J'}_{as}}_{g^J_{as}}$は,核スピンの対称性が基底準位と発光上準位で同じなら1,異なれば0となる.

\subsection{基底準位の振動・回転状態占有率}
発光上準位と同様に,基底準位の振動・回転状態占有数にもボルツマン分布を仮定する.
振動温度を$T^X_{\rm vib}$,回転温度を$T^{Xv}_{\rm rot}$として,$n_{XvJ}$は
\begin{equation}
\begin{split}
    n_{XvJ} &= \frac{ (2J+1) g^J_{\rm as} \exp \left( - \frac{E^{Xv}_{\rm rot}(J)}{k_{\rm B} T^{Xv}_{\rm rot}} \right) \exp \left( - \frac{E^{X}_{\rm vib}(v)}{k_{\rm B} T^{X}_{\rm vib}} \right) }{\sum_{vJ} (2J+1) g^J_{\rm as} \exp \left( - \frac{E^{Xv}_{\rm rot}(J)}{k_{\rm B} T^{Xv}_{\rm rot}} \right) \exp \left( - \frac{E^{X}_{\rm vib}(v)}{k_{\rm B} T^{X}_{\rm vib}} \right)}\\
    &= C (2J+1) g^J_{\rm as} \exp \left( - \frac{E^{Xv}_{\rm rot}(J)}{k_{\rm B} T^{Xv}_{\rm rot}} \right) \exp \left( - \frac{E^{X}_{\rm vib}(v)}{k_{\rm B} T^{X}_{\rm vib}} \right)
\end{split}
\end{equation}
と書ける.
ここで,$(2J+1)$は回転の縮退度である.
$g^J_{\rm as}$は核スピンの縮退度であり,$J$が奇数の時3,偶数の時1となる.
また,$C = 1 / \sum_{vJ} (2J+1) g^J_{\rm as} \exp \left( - \frac{E^{Xv}_{\rm rot}(J)}{k_{\rm B} T^{Xv}_{\rm rot}} \right) \exp \left( - \frac{E^{X}_{\rm vib}(v)}{k_{\rm B} T^{X}_{\rm vib}} \right)$である.
式 (2.12)のコロナモデルの式に代入すると,次の式が導ける.
\begin{equation}
    \frac{n_{dv'J'}}{\left( 2J'+1 \right) g^{J'}_{\rm as}} = C' \frac{ \sum_{v, J} \left[ R^{dv'J'}_{XvJ} (2J+1) g^J_{\rm as} \exp \left( - \frac{E^{Xv}_{\rm rot}(J)}{k_{\rm B} T^{Xv}_{\rm rot}} \right) \exp \left( - \frac{E^{X}_{\rm vib}(v)}{k_{\rm B} T^{X}_{\rm vib}} \right) \right] } {\left( 2J'+1 \right) g^{J'}_{\rm as} \sum_{v''J''} A^{dv'J'}_{av''J''} }
\end{equation}
ただし,$v', J'$に依存しない定数を$C'$にまとめた.
また,右辺の$T^{Xv}_{\rm rot}$について,回転温度は回転定数に比例するという関係式\cite{rot-temperature-ratio}
\begin{equation}
    T^{Xv}_{\rm rot} = \frac{B^{Xv}}{B^{dv'}} T^{dv'}_{\rm rot}
\end{equation}
を用いて,2.2.2章で求めた発光上準位の回転温度から計算できる.
式 (2.5)によって求めた$n_{dv'J'}$を式 (2.17)に代入し,右辺の$T^{X}_{\rm vib}$をフィッティングによって求めることができる.

以上の手順で求まった基底準位の振動・回転温度を式 (2.16)に代入することにより,基底準位の振動・回転状態占有率を求めることができる.