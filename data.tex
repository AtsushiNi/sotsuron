\chapter{実験データ}

\section{実験データ}
図\ref{fig:picture-example}に,CCDカメラで撮影した画像の例を示す.
図\ref{fig:spectrum-example}は画像の各ピクセルの値を縦に合計したものをプロットしたグラフである.
また,アノードに電流を流したまま電極間の電流を止めてプラズマを消した状態で同様にデータを取ったものが図\ref{fig:back-spectrum-example}である.
横方向のピクセルが1027のデータが不自然なことが見て分かるが,これはCCDカメラのある一つのピクセルに異常があり,光の大きさに関わらず異常な値を出力していることが原因である.
プラズマをつけて得たデータからプラズマを消したバックグラウンドのデータを減算することで,このようなノイズを打ち消すことができる.
バックグラウンドのデータを除いたものが図\ref{fig:true-spectrum-example}である.
以降の解析では,全ての画像に対してこの処理を施したものを使用した.

\section{波長校正}
横方向のピクセルを波長に変換するために,波長校正を行った.
データベース\cite{H2-spectrum-data}からFulcher-α帯発光線の波長を取得した.
そのデータを,発光線の中心となるピクセルに対してプロットし,二次関数でフィッティングすることで,ピクセルと波長の対応関係を得た(図\ref{fig:pixel-to-wavelength}).
波長校正後のスペクトルを図\ref{fig:all-spectrum}に示す.
なお,($N', N''$) = (1, 1), (2, 2), ...の発光線を順にQ1, Q2, ...としている.