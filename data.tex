\chapter{発光スペクトルの計測}
\section{実験の手順}
以下の手順に沿って発光スペクトルを計測した.
\begin{enumerate}
    \item 流量調節バルブを閉じ切って圧力が$10^{-6}$ Torr程度に下がっていることを確認した
    \item 流量調節バルブを少しずつ開き,圧力が$10^{-3}$ Torr程度になるまで水素を注入した.
    \item カソードに電圧を印加し,電流が20 A程度になるように調節した.
    \item アノード,カソード間に電圧を印加し,プラズマを生成した.プラズマ電流が2.5 Aになるように調節した.
    \item CCDカメラでスペクトルの撮影を行った.ステッピングモーターを回して撮影する波長帯を変えながら複数の画像を得た.
    \item アノード,カソード間の電源を切ってプラズマを消した後,同様に撮影を行った.
\end{enumerate}

\section{実験データ}
図\ref{fig:picture-example}に,上述の手順5で撮影した画像の例を示す.
図\ref{fig:spectrum-example}は図\ref{fig:picture-example}の各画素のカウント数を縦に合計したものをプロットしたグラフである.
また,手順6で得たデータのうち,図\ref{fig:spectrum-example}と波長範囲が同じものを図\ref{fig:back-spectrum-example}に示す.
横方向のピクセルが1027のデータが不自然なことが見て分かるが,これはCCDカメラのある一つのピクセルが,光の強度に関わらず異常な値を出力していることが原因である.
プラズマをつけて得たデータからプラズマを消した状態で得たバックグラウンドのデータを減算することで,このようなノイズを打ち消すことができる.
図\ref{fig:spectrum-example}からバックグラウンドのデータを除いたものが図\ref{fig:true-spectrum-example}である.
以降の解析では,全ての画像に対してこの処理を施したものを使用した.

\section{波長校正}
横方向のピクセルを波長に変換するために,波長校正を行った.
まず,光の強度が最大となるピクセルを発光線の中心とし,その中心に対してデータベース\cite{H2-spectrum-data}から取得したFulcher-α帯発光線の波長をプロットした.
さらに,二次関数でフィッティングすることで,ピクセルと波長の対応関係を得た(図\ref{fig:pixel-to-wavelength}).
得られた対応関係からピクセルを波長に変換した.
波長校正後のスペクトルを図\ref{fig:all-spectrum}に示す.
なお,($J', J''$) = (1, 1), (2, 2), ...の発光線を順にQ1, Q2, ...としている.