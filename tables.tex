\listoftables
\addcontentsline{toc}{chapter}{表目次}

\begin{table}
    \caption{水素分子の分子定数\cite{nist}}
    \label{table:molecular-constants}
    \centering
    \begin{tabular}{cccccc}
        \hline
        準位 & $D_{\rm e}$ ($\rm cm^{-1}$)& $B_{\rm e}$ ($\rm cm^{-1}$)& $\alpha_{\rm e}$ ($\rm cm^{-1}$)& $\overline{\omega_{\rm e}}$ ($\rm cm^{-1}$)& $\overline{\omega_{\rm e}} \chi_{\rm e}$ ($\rm cm^{-1}$)\\
        \hline
        $X^1 \Sigma^+_g$(基底準位) & 0.0471 & 60.853 & 3.062 & 4401.21 & 121.33\\
        $d^3 \Pi^-_u$(発光上準位) & 0.0191 & 30.364 & 1.545 & 2371.57 & 66.27\\
        $a^3 \Sigma^+_g$(発光下準位) & 0.0216 & 34.216 & 1.671 & 2664.83 & 71.65\\
        \hline
    \end{tabular}
\end{table}

\begin{table}
    \caption{発光上準位の回転温度}
    \label{table:upper-temperatures}
    \centering
    \begin{tabular}{ccc}
        \hline
        振動準位 & $T^{dv'}_{\rm rot}$ (K)\\
        \hline
        $v$ = 0 & 310\\
        $v$ = 1 & 280\\
        $v$ = 2 & 270\\
        \hline
    \end{tabular}
\end{table}

\begin{table}
    \caption{基底準位の振動・回転温度}
    \label{table:fitting-result}
    \centering
    \begin{tabular}{ccc}
        \hline
        振動準位 & $T^{Xv}_{\rm rot}$ (K) & $T^{X}_{\rm vib}$ (K)\\
        \hline
        $v$ = 0 & 620 & 4150\\
        $v$ = 1 & 570 & 4150\\
        $v$ = 2 & 530 & 4150\\
        \hline
    \end{tabular}
\end{table}

\begin{table}
    \caption{本研究対象のプラズマとLHD周辺領域のプラズマ\cite{ishihara}との違い}
    \label{table:LHD-and-this-plasma}
    \centering
    \begin{tabular}{ccc}
        \hline
         & 本研究対象のプラズマ & LHD周辺領域のプラズマ\\
        \hline
        電子温度 & 7 eV & 15 eV\\
        電子密度 & $10^{16}$ $\rm m^{-3}$ & $10^{18}$ $\rm m^{-3}$\\
        \hline
    \end{tabular}
\end{table}

\begin{table}
    \caption{本研究対象のプラズマとLHD周辺領域のプラズマ\cite{ishihara}における基底準位の振動・回転温度}
    \label{table:ground-result-compare}
    \centering
    \begin{tabular}{ccc}
        \hline
        & 本研究対象のプラズマ & LHD周辺領域のプラズマ\\
        \hline
        $T^X_{\rm vib}$ (K)& 4150 & 9200\\
        $T^{Xv}_{\rm rot}$($v$ = 0) (K)& 620 & 480\\
        $T^{Xv}_{\rm rot}$($v$ = 1) (K)& 570 & 450\\
        $T^{Xv}_{\rm rot}$($v$ = 2) (K)& 530 & 430\\
        \hline
    \end{tabular}
\end{table}
