\chapter{結言}

熱陰極グロー放電プラズマにおいて,ツェルニ・ターナー型分光器を用いて600-630 nmの波長域の水素分子の発光スペクトルを計測した.
データベースから得た発光線の波長データとスペクトルを比較することによって,波長校正を行い,発光線の特定を行った.
得られたスペクトルを歪正規分布関数でフィッティングし,面積を計算することで,発光強度を求めた.
さらに,発光強度から発光上準位の振動・回転状態占有率を計算し,回転エネルギーに対するボルツマンプロットを作成した.
振動準位$v'=0$では2温度のボルツマン分布が確認できたが,$v'=1,2$では計測データが足りず確認できなかった.
$v'=0$の振動・回転状態占有率について,2温度のボルツマン分布の式でフィッティングを行い,低回転温度,高回転温度,高回転温度を持つ振動・回転状態占有数の比率を求めた.
また,LHDのデータと比較すると,低回転温度は同程度であったが高回転温度はLHDの約1.4 倍であることが分かった.
次に基底準位に関する解析を行った.
基底準位の$J \leq 5$の範囲において振動・回転状態占有率にボルツマン分布を仮定した.
発光上準位の占有数に関する方程式から導出されるコロナモデルを適用し,基底準位の振動・回転温度をフィッティングにより求めた.
$v=0,1$の回転温度は精度よく求まったが,$v=2$の回転温度はフィッティングの誤差が大きく,精確に求めることはできなかった.
基底準位の回転温度と振動温度をボルツマン分布の式に代入することで,基底準位の振動・回転状態占有率を求め,回転エネルギーに対するボルツマンプロットを示した.
また,LHDのデータと比較を行った.
$v=0,1$の回転温度はLHDと同程度であり,振動温度が回転温度に比べてはるかに高いという点でもLHDと共通していることが分かった.
