\chapter{結言}

熱陰極グロー放電プラズマにおいて,ツェルニ・ターナー型分光器を用いて600-630 nmの波長域の水素分子の発光スペクトルを計測した.
データベースから得た発光線の波長データとスペクトルを比較することによって,波長校正を行い,発光線の特定を行った.
Fulcher-α帯Q枝スペクトルとして特定できた発光線は17本であった.
得られたスペクトルを歪正規分布関数でフィッティングし,面積を計算することで,発光強度を求めた.
さらに,発光強度から発光上準位の振動・回転状態占有率を計算した.
計算した占有率の,回転エネルギーに対するボルツマンプロットを作成すると,振動量子数が5以下の領域ではデータが直線状に並んだ.
発光上準位の占有率がボルツマン分布に従うと仮定し,ボルツマンプロットを直線でフィッティングすることにより,その傾きから発光上準位の回転温度を求めた.
回転温度は265-310 Kであった.
発光上準位の占有数に関する方程式から導出されるコロナモデルを適用した.
さらに基底準位の振動・回転状態占有率にボルツマン分布を仮定することで,基底準位の振動温度をフィッティングにより求めた.
基底準位の振動温度は4146 Kであった.
さらに,発光上準位の回転温度と基底準位の回転温度の関係式により,基底準位の回転温度は532-621 Kであると求まった.
基底準位の回転温度と振動温度をボルツマン分布の式に代入することで,基底準位の振動・回転状態占有率を求め,回転エネルギーに対してプロットした.
また,回転量子数が6以上の領域で発光上準位における占有率のボルツマンプロットが直線から外れることから,水素分子の回転温度が視線方向に対して一定ではないことが考えられた.