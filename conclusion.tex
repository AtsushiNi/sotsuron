\chapter{結言}

熱陰極グロー放電プラズマにおいて,ツェルニ・ターナー型分光器を用いて600-630 nmの波長域の水素分子の発光スペクトルを計測した.
データベースから得た発光線の波長データとスペクトルを比較することによって,波長校正を行い,発光線の特定を行った.
得られたスペクトルを歪正規分布関数でフィッティングし,面積を計算することで,発光強度を求めた.
さらに,発光強度から発光上準位の振動・回転状態占有率を計算した.
計算した占有率の,回転エネルギーに対するボルツマンプロットを作成すると,振動量子数が5以下の領域ではデータが直線状に並んだ.
発光上準位の占有率がボルツマン分布に従うと仮定し,ボルツマンプロットをフィッティングすることにより,発光上準位の回転温度を求めた.
回転温度は270-310 Kであった.
次に基底準位に関する解析を行った.
基底準位の振動・回転状態占有率にボルツマン分布を仮定した.
発光上準位の回転温度と基底準位の回転温度の関係式により,基底準位の回転温度は530-620 Kであると求まった.
発光上準位の占有数に関する方程式から導出されるコロナモデルを適用し,基底準位の振動温度をフィッティングにより求めた.
基底準位の振動温度は4150 Kであった.
基底準位の回転温度と振動温度をボルツマン分布の式に代入することで,基底準位の振動・回転状態占有率を求め,回転エネルギーに対するボルツマンプロットを示した.
また,本研究対象のプラズマと種類の異なるプラズマであるLHDで行われた計測結果との比較を行った.
基底準位の回転温度に比べ基底準位の振動温度が,本研究対象のプラズマでは7.2 倍,LHDでは20.3 倍と共にはるかに高いことが分かった.
さらに,基底準位において振動準位が大きくなるにつれて回転温度が低くなる点でも2つのプラズマは共通しているということが分かった.